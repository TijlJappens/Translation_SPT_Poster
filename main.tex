\documentclass[
    margin=1in,
    innermargin=-4.5in,
    ]{tikzposter}

% Choose size here
\geometry{paperwidth=33.11in,paperheight=46.81in} %A0
% \geometry{paperheight=33.11in,paperwidth=23.4in} %A1
    
\usepackage[utf8]{inputenc}
\usepackage{csquotes}
\usepackage{amsmath}
\usepackage{amsfonts}
\usepackage{amsthm}
\usepackage{amssymb}
\usepackage{mathrsfs}
\usepackage{graphicx}
\usepackage{lipsum}
\usepackage[export]{adjustbox}
\usepackage{tcolorbox}
\usepackage[font=small,labelfont=bf]{caption} % Required for specifying captions to tables and figures
\usepackage{enumitem}
\usepackage[backend=biber,style=numeric]{biblatex}
\usepackage{glasgow-poster-theme}
\usepackage{physics}
\usepackage{relsize}
\usepackage{caption}
\makeatletter
\setlength{\TP@visibletextwidth}{31.0in}
\setlength{ \TP@visibletextheight}{45in}
\makeatother
\usepackage{bm}
\usepackage{bbm}

\addbibresource{refs.bib}

% set theme parameters
\tikzposterlatexaffectionproofoff
\usetheme{UniGlasgowTheme}
\usecolorstyle{UniGlasgowStyle}

\usepackage[scaled]{helvet}
\renewcommand\familydefault{\sfdefault} 
\renewcommand{\vec}[1]{\bm{#1}}
\renewcommand{\AA}{\mathcal{A}}
\newcommand{\BB}{\mathcal{B}}
\newcommand{\ZZ}{\mathbb{Z}}
\newcommand{\CC}{\mathbb{C}}
\newcommand{\NN}{\mathbb{N}}
\newcommand{\RR}{\mathbb{R}}
\newcommand{\Bf}{\mathfrak{B}}
\renewcommand{\d}{\text{d}}
\usepackage[T1]{fontenc}


\title{\parbox{0.8\linewidth}{\Huge \textbf{SPT indices emerging from translation invariance in two dimensional quantum spin systems}}}
\author{Tijl Jappens\textsuperscript{1}}
\institute{\textsuperscript{1}Instituut voor theoretische fysica, Katholieke Universiteit Leuven}

% Adjust trim if title doesn't have two lines
\titlegraphic{\includegraphics[width=0.35\linewidth, clip]{figures/KU_Leuven_logo.png}}


% begin document
\begin{document}
\maketitle
\centering
\begin{columns}
    \column{0.5}
    \block{Setup}{
       \textbf{An operator algebra} is constructed using:\\
	\begin{tabular}{ l l }
		$\bullet$ A lattice $\Lambda$ ($\ZZ$ or $\ZZ^2$).&$\bullet$An on site Hilbert space $\CC^d$.\\
		$\bullet$ For each $\Gamma\in \Bf_\Lambda$ (the set of finite subsets of $\Lambda$),&$\bullet$The local algebra \\
		\phantom{$\bullet$}a finite dimensional algebra $\AA_\Gamma=\bigotimes_{i\in\Gamma}\BB(\CC^d)$.&$\qquad \AA_{\text{loc}}=\bigcup_{\Gamma\in\Bf_\Lambda}\AA_{\Gamma}$.
	\end{tabular}\\
	The operator algebra is the norm closure of the local algebra $\AA_{\Lambda}=\overline{\AA_{\text{loc}}}$.\\\\
	\textbf{A group action} is constructed using:\\
	\begin{tabular}{ l l }
		$\bullet$ A group $G$.$\qquad$&$\bullet$ A representation: $U\in\textrm{Hom}(G,U(\CC^d))$.
	\end{tabular}\\
	$\bullet$ For each $\Gamma\in\Bf_\Lambda$, a local group action: $\beta_\Gamma(g)=\textrm{Ad}(\bigotimes_{i\in\Gamma}U(g))$\\
	The group action $\beta\in\textrm{Hom}(G,\textrm{Aut}(\AA))$ is defined as $\beta:=\lim_{\Gamma\rightarrow \Lambda}\beta_\Gamma$.\\\\
	\textbf{The automorphism translating the lattice to the right} will be called $\gamma$.\\\\
	\textbf{A bounded interaction} is:\\
	$\bullet$ A map $\Phi:\Gamma\in\Bf_\Lambda\rightarrow \Phi_\Gamma\in\AA_\Gamma$ where each $\Phi_\Gamma$ is hermitian\\
	$\bullet$ such that there exists a monotomically decreasing positive function $F:\NN\rightarrow\RR^+$\\
	\phantom{$\bullet$} decreasing faster then any polynomial.\\
	$\bullet$ such that
	\[\norm{\Phi}_F:=\sup_{x,y\in\Lambda}\frac{1}{F(\abs{x-y})}\sum_{\Gamma\in\Bf_{\Lambda},\Gamma\ni x,y}\norm{\Phi_\Gamma}<\infty.\]
	$\bullet$ $\Phi$ is called $G$-invariant (translation invariant) if $\forall \Gamma\in\Bf_\Lambda,$
	\begin{align*}
		\beta(g)(\Phi_\Gamma)&=\Phi_\Gamma&(\gamma(\Phi_\Gamma)&=\Phi_{\gamma(\Gamma)}).
	\end{align*}
	\textbf{A locally generated automorphism (LGA)} is constructed using a one parameter family of bounded interactions. It is the solution of the ode ($\forall A\in\AA_\Lambda$)
	\[\frac{\d}{\d \lambda}\alpha_\lambda(A)=-i\sum_{\Gamma\in\Bf_\Lambda}\alpha_\lambda([\Phi_\Gamma(\lambda),A]).\]
	\textbf{A state} $\omega:\AA_\Lambda\rightarrow \CC$\\
	$\bullet$ is short range entangled (SRE) if there exists an LGA such that $\omega\circ\alpha_1$ is a product state.\\
	$\bullet$ is $G$-invariant (translation invariant) if
	\begin{align*}
		\omega\circ\beta(g)&=\omega&(\omega\circ\gamma&=\omega).
	\end{align*}
	\textbf{Two states $\omega_1$ and $\omega_2$ are called $G$($\times$translation-)equivalent} if there exists an LGA ($\alpha$) generated by a $G$-invariant (translation invariant) interaction such that $\omega_1\circ\alpha_1=\omega_2$.\\\\
	\textbf{There is an index} namely:\\
	$\bullet$ $G$-invariant states over the trivial lattice (with one site) carry an $H^1(G,U(1))$-valued index.\\
	$\bullet$ $G$-invariant states over $\ZZ$ carry an $H^2(G,U(1))$-valued index \cite{ogata2019classification}\cite{kapustin2021classification}.\\
	$\bullet$ $G$-invariant SRE states over $\ZZ^2$ carry an $H^3(G,U(1))$-valued index \cite{ogata2021h3gmathbb}.\\
	$\bullet$ If two states carry different indices they cannot be $G$-equivalent.
    }
    
    \block{Example}{
	\begin{center}
		\begin{tikzpicture}[scale=3]
			\fill[fill=blue] (-2.2,2.5) rectangle (-1.8,-2.5);
			\foreach \j in {-2,...,2}{
       			    \fill[fill=black] (-2,\j) circle(3pt);
        		 };
   		\node[] at (-2, -3.2)   (a) {$\mathlarger{\mathlarger{\mathlarger{\psi}}}$};
    		\node[] at (-1, 0)   (b) {$\mathlarger{\mathlarger{\mathlarger{\rightarrow}}}$};
   		\node[] at (3.5, -3.2)   (c) {$\mathlarger{\mathlarger{\mathlarger{\omega=\bigotimes_{i\in\ZZ}\psi}}}$};
    		\foreach \i in {0,...,7}{
			\fill[fill=blue] (\i-0.2,2.5) rectangle (\i+0.2,-2.5);
    		};
    
    		% loop over the lattice points
    		\foreach \i in {0,...,7}
      		\foreach \j in {-2,...,2}{
        		\fill[fill=black] (\i,\j) circle(3pt);
      		};
		\end{tikzpicture}
	\end{center}

	\captionof{figure}{\label{fig:1}This displays a way to construct a translation invariant state on the lattice $\ZZ^2$ using a state on the lattice $\ZZ$.}
	$\:$\\
	$\bullet$ Let $\psi_1$ and $\psi_2$ be two $G$-invariant SRE states with different $H^2(G,U(1))$-valued indices.\\
	$\bullet$ Let $\omega_1=\bigotimes_{i\in\ZZ}\psi_1$ and $\omega_2=\bigotimes_{i\in\ZZ}\psi_2$ be the $G$-invariant, translation invariant SRE states over the lattice $\ZZ^2$ constructed in figure \ref{fig:1}.\\
	$\bullet$ Can $\omega_1$ and $\omega_2$ ever be $G\times$translation-equivalent?\\\\
	It turns out the answer is no because they carry different translation indices.
    }


    \column{0.5}
    \block{Results}{
	\textbf{Result 1:} $G$-invariant, translation invariant SRE states over the one dimensional lattice $\ZZ$ carry an $H^2(G,U(1))\oplus H^1(G,U(1))$-valued index.\\\\
	\textbf{Result 2:} $G$-invariant SRE states over the two dimensional lattice $\ZZ^2$ with one translation symmetry carry an $H^3(G,U(1))\oplus H^2(G,U(1))$-valued index.\\\\
	\textbf{Result 3:} $G$-invariant SRE states over the two dimensional lattice $\ZZ^2$ with two translation symmetries carry an $H^3(G,U(1))\oplus H^2(G,U(1))\oplus H^2(G,U(1))\oplus H^1(G,U(1))$-valued index.
    }

    \block{Conclusions}{
    }
    
    \block{References}{
                \begin{center}
                   \mbox{}\vspace{-1\baselineskip}
    \printbibliography[heading=none] 
        \end{center}
        }

\end{columns}
\end{document}